\documentclass{article}

\usepackage{lipsum}
\usepackage{comment}
\usepackage[margin = 1in, includefoot]{geometry}

%Header and Footer stuff
\usepackage{fancyhdr}
\pagestyle{fancy}
\fancyhead{}
\fancyfoot{}
\fancyfoot[R]{\thepage\ }
\renewcommand{\headrulewidth}{0pt}
\renewcommand{\footrulewidth}{1pt}

%Header and Footer stuff over

\begin{document}

%commenting using package
\begin{comment}
this is used to comment out lines.
\title{Robust Anti-Spoofing Algorithm}
\author{Innovative Coder}
\date{August 27, 2017}
\maketitle
above written code will print the report in very lame format 
the code written below is improved.
\end{comment}
%commenting over
\begin{titlepage}
	\begin{center}
	\line(1,0){300}\\
	[0.25 in]
	\huge{\bfseries Anti-Spoofing Algorithm}\\
	[2mm]
	\line(1,0){200}\\
	[1.5mm]
	\textsc{\LARGE Northern India engineering college}\\
	[0.75mm]
	\textsc{\Large -Affiliated to GGSIPU}\\
	[12.5cm]
	\end{center}
	
	\begin{flushright}
	\textsc{\large Jasneet Singh Sawhney\\
	\#IEEE member ID - 94379222\\
	August 27, 2017\\}
	\end{flushright}

	
\end{titlepage}
\section{Introduction}\label{sec:intro}
This is the Robust Anti Spoofing Algorithm implementation using various algorithms.

The code is well explained.
\lipsum[1]

\newpage
\section{LBP : Local Binary Pattern}
LBP is an intelligent way of checking the illumination. 

The intorduction of this algorithm is found on page \pageref{sec:intro}
\subsection{How it works?}
The pattern is described by an illumination matrix which is  taken as relative.
\subsubsection{It contains the image of the working of LBPs}








\end{document}